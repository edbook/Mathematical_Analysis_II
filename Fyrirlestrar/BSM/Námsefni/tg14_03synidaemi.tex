%%%%%%%%%%%%%%%%%%%%%%%%%%%%%%%%%%%%%%%%%%%%%%%%%%%%%%%%%%%%%%%%%%%%%
% LEIÐBEININGAR:
% Skýrið þessa skrá tg14_dX-X.X.X.tex þar sem
% fyrsta Xið er númer dæmatímans sem þið mætið í og hin Xin eru númer
% dæmisins sem þið leysið, þ.e. kafli.undirkafli.númer
% 
% Leiðbeiningar fyrir LaTeX getið þið fundið í möppunni Kennsluáætlun
% skráin latex_introduction.pdf inniheldur flest allt sem þið þurfið
% að vita um LaTeX.
%%%%%%%%%%%%%%%%%%%%%%%%%%%%%%%%%%%%%%%%%%%%%%%%%%%%%%%%%%%%%%%%%%%%%

\documentclass[14pt,a4paper]{article}
\usepackage[utf8]{inputenc} 
\usepackage[english,icelandic]{babel}
\usepackage{t1enc}
\usepackage{hyperref}
\usepackage{amsmath}
\usepackage{amsfonts}
\usepackage{verbatim}
\usepackage{amssymb}
\usepackage{mdwlist} 
\renewcommand{\epsilon}{\varepsilon}

\voffset=-1.0in
\hoffset=-0.3in
\textwidth=6in
\textheight=10.2in

\begin{document}
\noindent Háskóli Íslands \hfill vor 2014\\
Töluleg greining STÆ405G \hfill kafli 3.1
\rule[8pt]{\textwidth}{1pt}


\bigskip

\begin{center}
{\large Sýnidæmi}
\end{center}

% Skrifið upp fyrirmælin í dæminu í slaufusviganna
\emph{Þetta er dæmi um jöfnuhneppi þar sem nauðsynlegt er að nota vendingu til þess
að fá nothæfa lausn. Við sjáum að það skiptir því miklu máli hvernig við útfærum 
Gauss-eyðinguna.
Jöfnuhneppið sem við ætlum að skoða er }
$$\left[
\begin{array}{ll}
\epsilon & 1 \\
1 & 1 
\end{array}\right]
\left[
\begin{array}{l}
x_1\\
x_2
\end{array}\right]
= \left[\begin{array}{l}
1\\
2
\end{array}\right]
$$

\medskip

% Hér byrjar svo lausnin ykkar
\subsection*{Nákvæm lausn}
Byrjum á því að finna nákvæma lausn með Gauss-eyðingu.
Margföldum fyrstu línuna með $\frac 1\epsilon$ og drögum frá
annarri línunni.
$$
\left[
\begin{array}{ll|l}
\epsilon & 1 & 1\\
1 & 1 & 2
\end{array}\right]
\sim
\left[
\begin{array}{ll|l}
\epsilon & 1 & 1\\
0 & 1-\frac 1\epsilon & 2 -\frac 1\epsilon
\end{array}\right]
$$
Af þessu sjáum við að 
$x_2 = \frac{2-\frac 1\epsilon}{1-\frac 1\epsilon} = 1 - \frac{\epsilon}{1-\epsilon}$,
og þá er 
$x_1 = \frac{1-1 x_2}{\epsilon} = 1+ \frac{\epsilon}{1-\epsilon}$.

\subsection*{Lausn í tölvu, án vendingar}
Athugum hvernig sömu reikningar eru framkvæmdir í tölvu, og gerum ráð 
fyrir að $\epsilon$ sé minna heldur en nákvæmnin í tölvunni. Það þýðir að tölvan
getur ekki gert greinarmun á tölunum $1-\frac 1\epsilon$ og $-\frac 1\epsilon$, og eins á 
tölunum $2-\frac 1\epsilon$ og $-\frac 1\epsilon$, því $\frac 1\epsilon$ er svo miklu
stærra en $1$ og $2$. Gauss-eyðing út frá fyrstu línu skilar eins og áður
$$
\left[
\begin{array}{ll|l}
\epsilon & 1 & 1\\
1 & 1 & 2
\end{array}\right]
\sim
\left[
\begin{array}{ll|l}
\epsilon & 1 & 1\\
0 & 1-\frac 1\epsilon & 2 -\frac 1\epsilon
\end{array}\right]
$$
En af ofangreindum ástæðum þá lítur tölvan á seinna hneppið sem
$$
\left[
\begin{array}{ll|l}
\epsilon & 1 & 1\\
0 & -\frac 1\epsilon & -\frac 1\epsilon
\end{array}\right]
$$
Þannig að $x_2 = \frac{-\frac 1\epsilon}{-\frac 1\epsilon} = 1$. 
Þá er $x_1 = \frac{1-1x_2}{\epsilon} = \frac{1-1}{\epsilon} = 0$.
Sem er langt frá réttu lausninni.

\subsection*{Lausn í tölvu, með vendingu}
Prófum að gera það sama, nema nú skulum við eyða út frá annarri línu.
Margföldum línu 2 með $\epsilon$ og drögum frá fyrstu línunni.
$$
\left[
\begin{array}{ll|l}
\epsilon & 1 & 1\\
1 & 1 & 2
\end{array}\right]
\sim
\left[
\begin{array}{ll|l}
0 & 1-\epsilon & 1-2\epsilon\\
1 & 1 & 2
\end{array}\right]
$$
Tölvan sér ekki mun á $1-\epsilon$ og 1, og eins á $1-2\epsilon$ og 1 þannig
að í hennar augum verður þetta að 
$$\left[
\begin{array}{ll|l}
0 & 1 & 1\\
1 & 1 & 2
\end{array}\right]
$$
Þannig að $x_2 = 1$, og þá fæst 
$x_1 = \frac{2-1x_2}{1} = 1$. Þetta er ekki hárrétt lausn, en mun betri en
sú sem við fengum áður.

\subsection*{Matlab}
Þið getið prófað að framkvæma þessa reikninga í Matlab. Það má gera með eftirfarandi
skipunum.
\begin{verbatim}
>>[ 1/5 1 ; 1 1 ]\[1 ; 2]
>>[ 1e-20 1 ; 1 1 ]\[ 1 ; 2]
\end{verbatim}
Í fyrri skipuninni þá er $\epsilon = \frac 15$, þá sést greinilega nákvæma lausnin að ofan.
En í seinni skipuninni þá er $\epsilon = 10^{-20}$ og þá lendum við í því að lausnin okkar
er ekki alveg rétt. Matlab er hins vegar það skynsamt að það beita skalaðri hlutvendingu
þannig að lausnin okkar verður ekki alveg út úr kú.

\bigskip
\begin{flushright}
 Benedikt Magnússon\\
 \today
\end{flushright}
\end{document}
