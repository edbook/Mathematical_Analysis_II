\lecture[10]{Kafli 10: Parabólskar hlutafleiðujöfnur}{lecture-text}
\date{13.~og 18.~apríl, 2014}

\begin{document}

\begin{frame}
	\maketitle
\end{frame}

\begin{frame}{Yfirlit}
\begin{block}{Kafli 10: Parbólskar hlutafleiðujöfnur}
\begin{center}
\begin{tabular}{|l|l|l|l|}\hline
Kafli &Heiti á viðfangsefni & Bls. & Glærur\\
\hline
10.0 & Almenn atriði um parabólskar hlutafleiðuj. & 803-804 & 3-6\\
10.1 & Hitajafnan með Dirichlet-jaðarskilyrði & 805-817 & \\
10.2 & Stöðugleiki & 818-827 & \\
10.3 & Almenn parabólsk verkefni & 828-840 & \\
10.4 & Blönduð jaðarskilyrði & 841-852 & \\ \hline
\end{tabular}
\end{center}
\end{block}
\end{frame}

\section*{10.0 Almenn atriði um jaðargildisverkefni}

\begin{frame}{10.0 Inngangur}
Eins og í kafla 9 þá ætlum við hér að skoða annars stigs línulegar 
hlutafleiðujöfnur á svæði $R$ í $\R^2$, \pause þetta eru jöfnur á forminu
\begin{multline}
A(x,y)\frac{\p^2 u}{\p x^2} + B(x,y)\frac{\p^2 u}{\p x\p y} + C(x,y)\frac{\p^2 u}{\p y^2} +\\
D(x,y)\frac{\p u}{\p x} + E(x,y)\frac{\p u}{\p y} + F(x,y)u = G(x,y).
\end{multline}
\begin{block}{Skilgreining}
 Annars stigs línuleg hlutafleiðujafna eins og að ofan kallast \emph{parabólsk}
 ef 
 $$
  A(x,y)C(x,y) - B(x,y)^2 = 0,
 $$
 fyrir öll $(x,y) \in R$.
\end{block}

\end{frame}


\begin{frame}{10.0 Parabólskar jöfnur}
 Við ætlum að láta aðra breytuna vera tímabreytu. 
 Þannig fæst tímaháð verkefni, en parabólskar jöfnur lýsa oft verkefnum þar 
 hlutur fer úr upphafsástandi í jafnvægisástand. \pause
 
 \begin{block}{Hagnýtingar}
 \begin{itemize}
  \item Dreifing hita og styrkleiki lausna
  \item Black-Scholes módelið í fjármálastærðfræði
  \item Brownhreyfing
 \end{itemize}
 \end{block}
 
 \pause
 
 \begin{block}{Hitajafnan}
 Við ætlum að einskorða okkur við mikilvægasta tilvikið, \emph{Hitajöfnuna}
 $$
  \frac{\p u}{\p t} =  D \Delta u,
 $$\pause
 í einni rúmbreytu þá þýðir þetta, \emph{Poisson jöfnuna}
 $$
  \frac{\p u}{\p t} =  D\frac{\p^2 u}{\p x^2}.
 $$
 \end{block}
\end{frame}

\begin{frame}{10.0 Jaðarskilyrði}

\begin{block}{Svæðið}
 Þar sem við höfum aðeins eina rúmbreytu þá er svæðið sem við skoðum 
 einfaldlega bil $[a,b]$
\end{block}
  
\begin{block}{Jaðarskilyrði}
 Eins og fyrir sporgerar hlutafleiðujöfnur (kafli 9) þá getum við haft þrjár gerðir
 af jaðarskilyrðum: Dirichlet,
 Neumann og  Robin.
\end{block}

\begin{block}{Verkefnið}
Við munum byrja á því að skoða Dirichlet-jaðarskilyrði: \pause 

Gefin föll
$u_a(t)$, $u_b(t)$ fyrir $t\geq 0$ og fall $f(x)$ á $[a,b]$ þá viljum
við finna fall $u(x,t)$ þannig að \pause
\begin{itemize}
 \item $\frac{\p u}{\p t} =  D \frac{\p^2 u}{\p x^2}$, \pause
 \item $u(a,t) = u_a(t)$ og $u(b,t) = u_b(t)$, \pause
 \item $u(x,0) = f(x)$. \pause
\end{itemize}
Talan $D$ kallast dreifnistuðull.
\end{block}

\end{frame}

\begin{frame}{10.0 Hitajafnan og þýð föll}
\begin{block}{Hitajafnan}
  $$
  \frac{\p u}{\p t} = D\  \Delta u,
 $$
 \end{block}
 
 \begin{block}{Athugasemd}
  Þegar hluturinn sem við erum að skoða er kominn í jafnvægisástand þá
  þýðir það að $\frac{\p u}{\p t} = 0$. \pause Þetta þýðir að 
  $$
     0 =  D\ \Delta u,
  $$
  það er,jafnvægisástandið uppfyllir meðalgildiseiginleikann (sjá glæru 9.6).
 \end{block}
 
\end{frame}

\section*{10.1 Hitajafnan --  Dirichlet-jaðarskilyrði }

\begin{frame}{10.1 Rúmbreytan gerð strjál }
 Byrjum á að skipta bilinu $[a,b]$ í $N$ hlutbil, hvert um sig með lengdina
 $h=(b-a)/N$:
 $$
    a = x_0 < x_1 < \ldots < x_{N-1} < x_N = b.
 $$ \pause
 
 Látum $v_j(t)$ vera nálgunargildið á $u(x_j,t)$.
 
 Með því að skipta annarri afleiðunni í 
 $$
 \frac{\p u}{\p t} =  D\ \Delta u,
 $$
 út fyrir samhverfan mismunakvóta þá fæst
 $$ 
 \frac{d v_j(t)}{d t} =  D\  \frac{v_{j-1}(t) -2 v_j(t) + v_{j+1}(t)}{h^2}, \qquad
 \text{fyrir } j=1,\ldots,N-1.
 $$
 Athugið að þetta er fyrsta stigs afleiðujafna eins og við fengumst við í kafla 7.
\end{frame}

\begin{frame}{10.1 Afleiðujöfnuhneppi stillt upp}
 Jöfnurhneppið 
 $$ 
 \frac{d v_j(t)}{d t} =  D\  \frac{v_{j-1}(t) -2 v_j(t) + v_{j+1}(t)}{h^2}, \qquad
 \text{fyrir } j=1,\ldots,N-1.
 $$ \pause
 með upphafsskilyrðunum $v_j(0) = f(x_j)$, \pause
 má svo skrifa
 $$ 
 \frac{d \vv_j(t)}{d t} =  -\frac{D}{h^2}\  (A\vv(t) + \bv(t)), \qquad \vv(0)=\fv
 $$
 þar sem 
 \begin{align*}
  \vv(t) &= [v_1(t) \ v_2(t) \ \cdots \ v_{N-1}(t) ]^T\\
  \fv &= [ f(x_1) \ f(x_2) \ \cdots \ f(x_{N-1}) ]^T\\
  \bv(t) &= [ -u_a(t) \ \ 0 \ \cdots \ 0 \ \ -u_b(t) ]^T \\
  A &=  \left[\begin{array}{cccccc}
2 & -1 &   &   &   &  \\
-1 & 2 & -1 &   &   &  \\
  & \cdot & \cdot & \cdot &   &  \\
  &   & \cdot & \cdot & \cdot &  \\
  &   &  & -1 & 2 & -1\\
  &   &   &   & -1 & 2
      \end{array}\right]
 \end{align*}
\end{frame}

\begin{frame}{10.1 Tímabreytan gerð strjál}
Til þess að leysa afleiðujöfnuhneppið hér á undan þá beytum við aðferðunum
úr 7.~kafla. Hægt er að nota hvort sem er aðferðir með fasta skrefastærð
(aðferð Eulars, Heun og Runge-Kutta) og aðferðir með breytilega skrefastærð
(RKF45 og RKV56). \pause

\medskip
Við munum notast við fastaskrefstærð $k$. Tímaskrefin eru því
$t_n = n k$, fyrir $n=0,1,2,\ldots,$. \pause

\medskip
Táknum nálgunargildið við 
$v_j(t_n)$ með $w_j^{(n)}$, og látum
$$
  \wv^{(n)} = [w_1^{(n)}(t) \ w_2^{(n)} \ \cdots \ w_{N-1}^{(n)} ]^T. 
$$

\end{frame}

\begin{frame}{10.1 FTCS -- Fram í tíma, miðsett í rúmi}
Skiptum afleiðunni í 
$$ 
 \frac{d \vv_j(t)}{d t} =  -\frac{D}{h^2}\  (A\vv(t) + \bv(t)), \qquad \wv^{(0)}=\fv
$$
út fyrir frammismun. Þá fæst
$$ 
 \frac{\wv^{(n+1)} - \wv^{(n)}}{ k} =  -\frac{D}{h^2}\  (A\wv^{(n)} + \bv(t_n)), 
 \qquad \wv^{(0)}=\fv.
$$\pause
Þetta er jafngilt því að heilda efstu jöfnuna frá $t=t_n$ upp í $t=t_{n+1}$ og
nálga heildið af hægri hliðinni með margfeldinu af gildinu í vinstri endapunktinum 
$t_n$ og billengdinni $k$. \pause

Þetta er aðferð Eulers úr kafla 7.2.
\end{frame}
 

\begin{frame}{10.1 FTCS -- Fram í tíma, miðsett í rúmi, frh.}
 Jöfnuhneppið
 $$ 
 \frac{\wv^{(n+1)} - \wv^{(n)}}{ k} =  -\frac{D}{h^2}\  (A\wv^{(n)} + \bv(t_n)), 
$$
má svo umrita sem
\begin{align*}
 \wv^{(n+1)} &= \wv^{(n)} -\lambda A\wv^{(n)} - \lambda \bv(t_n) \\
 & = (I-\lambda A) \wv^{(n)} - \lambda \bv(t_n),
\end{align*}
þar sem $\lambda = k\frac{D}{h^2}$. \pause
 
 Upphafsgildið $\wv^{(0)}$ er þekkt og hér er komin rakningarformúlu
 fyrir lausninni.  \pause
 
 \begin{block}{Skekkjan}
  Samhverfi mismunakvótinn í rúmbreytunni hefur skekkjuna
 $O(h^2)$ og frammismunurinn fyrir tímabreytuna hefur skekkjuna $O(k)$. 
 Skekkjan í þessari aðferð er því $O(h^2 + k)$.
 \end{block}
\end{frame}
 
\begin{frame}{10.1 BTCS - Aftur í tíma, miðsett í rúmi}
Ef við skiptum afleiðunni í 
$$ 
 \frac{d \vv_j(t)}{d t} =  -\frac{D}{h^2}\  (A\vv(t) + \bv(t)), \qquad \wv^{(0)}=\fv
$$
út fyrir bakmismun. Þá fæst
$$ 
 \frac{\wv^{(n+1)} - \wv^{(n)}}{ k} =  -\frac{D}{h^2}\  (A\wv^{(n+1)} + \bv(t_{n+1})), 
 \qquad \wv^{(0)}=\fv.
$$\pause
Þetta er jafngilt því að heilda efstu jöfnuna frá $t=t_n$ upp í $t=t_{n+1}$ og
nálga heildið af hægri hliðinni með margfeldinu af gildinu í hægri endapunktinum 
$t_{n+1}$ og billengdinni $k$. \pause

\end{frame}
 
\begin{frame}{10.1 BTCS - Aftur í tíma, miðsett í rúmi, frh.}
 Jöfnuhneppið
 $$ 
 \frac{\wv^{(n+1)} - \wv^{(n)}}{ k} =  -\frac{D}{h^2}\  (A\wv^{(n+1)} + \bv(t_{n+1})), 
$$
má svo umrita sem
\begin{align*}
 (I+\lambda A)\wv^{(n+1)} &= \wv^{(n)}  - \lambda \bv(t_{n+1}), 
\end{align*}
þar sem $\lambda = k\frac{D}{h^2}$. \pause
 
 Upphafsgildið $\wv^{(0)}$ er þekkt og hér er komin önnur rakningarformúla
 fyrir lausninni $\wv$.  \pause
 
 \begin{block}{Skekkjan}
  Samhverfi mismunakvótinn í rúmbreytunni hefur skekkjuna
 $O(h^2)$ og bakmismunurinn fyrir tímabreytuna hefur skekkjuna $O(k)$. 
 Skekkjan í þessari aðferð er því $O(h^2 + k)$.
 \end{block}
 
 \begin{block}{Fjöldi aðgerða}
  Þar sem $I+\lambda A$ er þríhornalínufylki þá þarf aðeins tvöfalt fleiri
  aðgerðir til þess að leysa þetta jöfnuhneppi heldur en þarf til þess að framkvæma margföldunina
  í FTCS-aðferðinni.
 \end{block}

 
 \end{frame}
 
\begin{frame}{10.1 Crank-Nicolson aðferð}
$$ 
 \frac{d \vv_j(t)}{d t} =  -\frac{D}{h^2}\  (A\vv(t) + \bv(t)),
$$

Í bæði FTCS og BTCS þá var skekkjan sem tilheyrði tímabreytunni $O(k)$.
Ef við viljum bæta þetta þá er einfaldast að nota Trapisuregluna
því skekkjan í henni er $O(k^2)$. \pause 

Svo ef við heildum jöfnuna að ofan frá $t=t_n$ upp í $t=t_{n+1}$ og
nálgum heildið af hægri hliðinni með meðaltalinu af gildunum í endapunktunum
margfaldað við billengdina
þá fæst
$$ 
 \wv^{(n+1)} - \wv^{(n)} =  -\frac{kD}{2h^2}\  
 \big( (A\wv^{(n)} + \bv(t_{n})) +  
 (A\wv^{(n+1)} + \bv(t_{n+1})) \big).
$$
\end{frame}
 
\begin{frame}{10.1 BTCS - Aftur í tíma, miðsett í rúmi, frh.}
 Jöfnuhneppið
 $$ 
 \wv^{(n+1)} - \wv^{(n)} =  -\frac{kD}{2h^2}\  
 \big( (A\wv^{(n)} + \bv(t_{n})) +  
 (A\wv^{(n+1)} + \bv(t_{n+1})) \big), 
$$
má svo umrita sem
\begin{align*}
 (I+\lambda A)\wv^{(n+1)} &= (I+\lambda A) \wv^{(n)}  - \lambda (\bv(t_{n}) 
 + \bv(t_{n+1})), 
\end{align*}
þar sem $\lambda = \frac{kD}{2h^2}$. \pause
 
 Upphafsgildið $\wv^{(0)}$ er þekkt og hér er því komin þriðja rakningarformúlun
 fyrir lausninni $\wv$.  \pause
 
 \begin{block}{Skekkjan}
  Samhverfi mismunakvótinn í rúmbreytunni hefur skekkjuna
 $O(h^2)$ og bakmismunurinn fyrir tímabreytuna hefur skekkjuna $O(k^2)$. 
 Skekkjan í þessari aðferð er því $O(h^2 + k^2)$.
 \end{block}
 
 \begin{block}{Fjöldi aðgerða}
  Fjöldi reikniaðgerða hér er u.þ.b.~summan af fjölda reikniaðgerða fyrir FTCS og BTCS.
 \end{block}

 
 \end{frame}
 
 \section*{10.2 Stöðugleiki}
 
 \begin{frame}{10.2 Stöðugleiki}
  Lausnin sem fæst úr aðferðunum hér á undan geta verið viðkvæmar fyrir skrefstærðinni $k$  sem við 
  veljum fyrir tímabreytuna. \pause
  
  
  Rifjum upp að aðferð kallast samleitin ef að nálgunarlausnin stefnir á 
  réttu lausnina þegar $t\to \infty$.
  
  \pause
  \begin{block}{Skilgreining}
    Við segjum að aðferð sé
    \begin{itemize}
     \item \emph{óskilyrt samleitin} ef aðferðin er samleitin fyrir öll $k$, \pause
     \item \emph{skilyrt samleitin} ef aðferðin er samleitin fyrir nógu lítil $k$,\pause
     \item \emph{óskilyrt ósamleitin} ef aðferðin er ósamleitin fyrir öll $k$.
    \end{itemize}

  \end{block}

 \end{frame}
 
 \begin{frame}{10.2 Stöðugleiki}
 Allar þrjár aðferðirnar sem  við skoðuðum hér á undan er hægt að setja
 fram sem 
 $$
  \wv^{(n+1)} = E\wv^{(n)} + \cv^{(n)},
 $$
 fyrir ákveðið fylki $E$ og vigra $\cv^{(n)}$.
 \pause
 \medskip
 
 Gefið $\wv^{(0)}$, þá þýðir þetta að
 \begin{align*}
  \wv^{(1)} &= E\wv^{(0)} + \cv^{(0)} \\
  \wv^{(2)} &= E\wv^{(1)} + \cv^{(1)} = E^2 \wv^{(0)} + (\cv^{(1)} + E\cv^{(0)}) \\
  &\vdots \\
  \wv^{(m)} &= E\wv^{(m-1)} + \cv^{(m-1)} = E^m \wv^{(0)} + \hat{\cv}, \\
  \end{align*}
  þar sem $\hat\cv = \c^{(m-1)} + E\cv^{(m-2)} + \ldots E^{m-1} \cv^{(0)}
  
  \end{frame}
\end{document}
  %
%   &-h^2 f_{1,2} + g_{0,2} +^2 \\
%   &-h^2 f_{2,2} \\
%   &\ldots \\
%   &-h^2 f_{N-2,2}\\
%   &-h^2 f_{N-1,2} + g_{N,2}\\
   & \ldots \ldots \ldots  & (\text{línur } k=2,\ldots,M-2)\\
  &-h^2 f_{1,M-1} + g_{1,M-1} + g_{0,M} & (\text{lína } k=M-1)\\
  &-h^2 f_{2,M-1} + g_{2,M} \\
  &\ldots \\
sddd  %&-h^2 f_{N-2,1} + g_{N-2,0} \\
  &-h^2 f_{N-1,M-1} + g_{N,M-1} + g_{N-1,M} ]\\
  \end{align*}}
\end{frame}



\section*{Fræðilegar spurningar}

\begin{frame}{Kafli 9: Fræðilegar spurningar}
\begin{enumerate}
  \item Hvað er átt við með því að lausn hlutafleiðujöfnu á svæði  $R$
    í plani 
    uppfylli {\it Dirichlet-jaðarskilyrði}?  \\
(Samheiti er {\it fallsjaðarskilyrði}.)
\item Hvernig er {\it útvísandi þverafleiða} $\partial u/\partial n$ af 
falli $u$ á svæði $R$ í plani skilgreind?  
\item Hvað er átt við með því að lausn hlutafleiðujöfnu á svæði $R$ í plani
    uppfylli {\it Neumann-jaðarskilyrði}? \\ (Samheiti eru {\it
      afleiðujaðarskilyrði}
og {\it flæðisjaðarskilyrði}.)
  \item Hvað er átt við með því að lausn hlutafleiðujöfnu á svæði  $R$
    uppfylli {\it Robin-jaðarskilyrði}?  \\ 
(Samheiti er {\it blandað jaðarskilyrði}.)
  \item Hvernig er  nálgunarjafna fyrir 
Poisson-jöfnu $\Delta u=f$ í innri skiptipunkti
í ferningslaga neti í plani leidd út? 
  \item Hvernig eru {\it felupunktur}  og {\it felugildi} notuð til
    þess að meðhöndla blandað jaðarskilyrði $\alpha_1 u+\alpha_2
    \partial u/\partial n=\alpha_3$ í jaðarpunkti svæðis $R$ í plani
    og hvernig verður nálgunarjafnan í þeim punkti? 
  \end{enumerate}
\end{frame}


\end{document}